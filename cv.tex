\documentclass[10pt,a4paper]{moderncv}
\usepackage{pdfpages}
\usepackage{import} \import{./}{setup.tex}

\name{Maximo}{Santoro}

\phone[mobile]{+54 911 6891 9279}
\email{msantoro@fi.uba.ar}
\extrainfo{
  \begin{tabular}{@{\hspace{2em}}c@{\hspace{2em}}}
    \href{https://www.linkedin.com/in/maximo-santoro/}{\faLinkedin \vspace{0.4mm} maximo-santoro} • \href{https://www.github.com/maximosantoro/}{ \faGithub \vspace{0.4mm} maximosantoro}
  \end{tabular}
}

\begin{document}

\spa{
    \begin{textblock*}{1.51cm}(19cm,0.2cm)
        \begin{shaded*}
        \centering
            \href{https://maximosantoro.github.io/cv/cv-es.pdf}{SPA}
        \end{shaded*}
    \end{textblock*}
}

\en{
    \begin{textblock*}{1.51cm}(19cm,0.2cm)
        \begin{shaded*}
        \centering
            \href{https://maximosantoro.github.io/cv/cv-en.pdf}{ENG}
        \end{shaded*}
    \end{textblock*}
}

\thispagestyle{onlyfooter}

\vspace{-3.5em}
\makecvtitle
\addtolength{\parskip}{6pt}

\vspace{-2em}
\section{\spa{Experiencia Laboral}\en{Work Experience}}


\cventry
    {\spa{March 2019 - Presente}\en{March 2019 - Now}}
    {\spa{Desarrollador Web}\en{Web Developer}}
    {\href{https://www.maximosantoro.com/}{Freelance}}{}{}
    {
        {\begin{itemize}
        \item \spa{Diseño, desarrollo y Mantenimiento de paginas web estaticas.} \en{Design, development and support for static web pages.}
    \end{itemize}}
    }


%\cventry
%    {\spa{Marzo 2019--Presente}\en{March 2019--Now}}
%    {\spa{Desarrollador de software}\en{Software developer}}
%    {\href{https://www.raiconet.com/}{Raico S.A.} \normalfont - \spa{Importación y exportación aérea y marítima.} \en{Air shipment and ocean freight services.}}{}{}
%    {\begin{itemize}
%        \item \spa{Mantenimiento y desarrollo de la aplicación mobile y de la aplicación web Raiconet} \en{Development and support of the web application and the mobile app of Raico S.A.}
%        \item \spa{Desarrollo de Exporta Simple, una plataforma web integrada con el Ministerio de Producción y Trabajo de Argentina} \en{Development of \textit{Exporta Simple}, a web platform integrated with the Argentine Ministry of Production}
%    \end{itemize}}

\cventry
    {\spa{Septiembre 2020 - Presente}\en{September 2020 - Now}}
    {\href{https://algoritmos9511.github.io/}{\spa{Colaborador - Algoritmos y Programación I - Curso Esaya}\en{Teaching assistant - Algorithms and Data Structures I}}}
    {Universidad de Buenos Aires, Facultad de Ingeniería}{}{}
    {}}
    \begin{itemize}
        \item \spa{Temas cubiertos en Algoritmos y Programación I: } \en{Covered topics in Algorithms and Programming I: }
    \end{itemize}}


\section{\spa{Educación}\en{Education}}

\cventry
    {\spa{2019 - Presente}\en{2019 - Now}}
    {\spa{Estudiante de Ingeniería en Informática y Electronica}\en{Software and Electronics Engineering student}}
    {Universidad de Buenos Aires, Facultad de Ingeniería}{}{}{}

\cventry
    {2013 - 2018}
    {\spa{Bachiller en Humanidades y Ciencias Sociales}\en{Bachelors degree in Humanities and Social Sciences}}
    {Colegio San Pablo}{}{}{}

\cventry
    {2018}
    {Cambridge in Advanced English (CAE)}
    {University of Cambridge}{}{}{\textit{Grade B}}

\section{\spa{Conocimientos}\en{Skills}}

\begin{itemize}
    \item \textbf{\spa{Lenguajes de programación}\en{Programming languages}:} C, JavaScript, Python, C\#/Unity, Godot/GDScript.

  %  \item \textbf{\spa{Paradigmas de programación y técnicas de diseño de algoritmos}\en{Programming paradigms and algorithm design techniques}:}
  %      \spa{Programación procedural, programación orientada a objetos, programación dinámica, división y conquista, metodologías greedy}\en{Procedural programming, object-oriented programming, dynamic programming, divide and conquer, greedy algorithms}

    \item \textbf{\spa{Otros lenguajes}\en{Other languages}:} HTML, CSS, \LaTeX.

    \item \textbf{\spa{Frameworks y otros}\en{Frameworks and miscellaneous}:} Git, GNU Make, Linux.
\end{itemize}

\IfFileExists{./notas.pdf}{\includepdf[pages=-]{notas.pdf}}{}
\end{document}
