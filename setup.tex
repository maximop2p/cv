\moderncvstyle{banking}
\moderncvcolor{grey} % blue, orange, green, red, purple, grey, black
\usepackage[utf8]{inputenc}
\usepackage[spanish,es-lcroman,es-notilde,english]{babel} 

% Setup de cambio de idiomas
\usepackage{ifthen}
\newif\ifspa
\newif\ifen
\newcommand{\spa}[1]{\ifspa#1\fi}
\newcommand{\en}[1]{\ifen#1\fi}

% Cambiar entre idiomas: Encapsular texto de cada idioma en las tags \spa{texto} o \en{text}
% Para usar en español, compilar normalmente. Para usar en ingles, pdflatex "\def\inenglish{1} \documentclass[10pt,a4paper]{moderncv}
\usepackage{pdfpages}
\usepackage{import} \import{./}{setup.tex}

\name{Maximo}{Santoro}

\phone[mobile]{+54 911 6891 9279}
\email{maximosantoro@protonmail.com}
\extrainfo{
  \begin{tabular}{@{\hspace{2em}}c@{\hspace{2em}}}
    \href{https://www.linkedin.com/in/maximo-santoro/}{\faLinkedin \vspace{0.4mm} maximo-santoro} • \href{https://www.github.com/maximosan/}{ \faGithub \vspace{0.4mm} maximosan}
  \end{tabular}
}

\begin{document}

\spa{
    \begin{textblock*}{1.51cm}(19cm,0.2cm)
        \begin{shaded*}
        \centering
            \href{https://maximosan.github.io/cv/cv-es.pdf}{SPA}
        \end{shaded*}
    \end{textblock*}
}

\en{
    \begin{textblock*}{1.51cm}(19cm,0.2cm)
        \begin{shaded*}
        \centering
            \href{https://maximosan.github.io/cv/cv-en.pdf}{ENG}
        \end{shaded*}
    \end{textblock*}
}

\thispagestyle{onlyfooter}

\vspace{-3.5em}
\makecvtitle
\addtolength{\parskip}{6pt}

\vspace{-2em}
\section{\spa{Experiencia Laboral}\en{Work Experience}}


\cventry
    {\spa{2019 - Presente}\en{2019 - Now}}
    {\spa{Desarrollador Web}\en{Web Developer}}
    {\href{https://www.maximosantoro.com/portfolio}{Freelance}}{}{}
    {
        {\begin{itemize}
        \item \spa{Diseño, desarrollo y mantenimiento de paginas web estáticas.} \en{Design, development and support for static web pages.}
    \end{itemize}}
    }


%\cventry
%    {\spa{Marzo 2019--Presente}\en{March 2019--Now}}
%    {\spa{Desarrollador de software}\en{Software developer}}
%    {\href{https://www.raiconet.com/}{Raico S.A.} \normalfont - \spa{Importación y exportación aérea y marítima.} \en{Air shipment and ocean freight services.}}{}{}
%    {\begin{itemize}
%        \item \spa{Mantenimiento y desarrollo de la aplicación mobile y de la aplicación web Raiconet} \en{Development and support of the web application and the mobile app of Raico S.A.}
%        \item \spa{Desarrollo de Exporta Simple, una plataforma web integrada con el Ministerio de Producción y Trabajo de Argentina} \en{Development of \textit{Exporta Simple}, a web platform integrated with the Argentine Ministry of Production}
%    \end{itemize}}

\cventry
    {\spa{2020 - Presente}\en{2020 - Now}}
    {\href{https://algoritmos9511.github.io/}{\spa{Colaborador - Algoritmos y Programación I - Curso Esaya}\en{Teaching assistant - Algorithms and Data Structures I}}}
    {Universidad de Buenos Aires, Facultad de Ingeniería}{}{}
    {
        \begin{itemize}
        \item \spa{Temas cubiertos: memoria dinámica, algoritmos básicos, estructuras de datos básicas. (C)} \en{Covered topics: dynamic memory, basic algorithms, basic data structures. (C)}
   \end{itemize}
   }
    


\section{\spa{Educación}\en{Education}}

\cventry
    {\spa{2019 - Presente}\en{2019 - Now}}
    {\spa{Estudiante de Ingeniería en Informática y Electrónica}\en{Software and Electronics Engineering student}}
    {Universidad de Buenos Aires, Facultad de Ingeniería}{}{}{}

\cventry
    {2013 - 2018}
    {\spa{Bachiller en Humanidades y Ciencias Sociales}\en{Bachelors degree in Humanities and Social Sciences}}
    {Colegio San Pablo}{}{}{}

\cventry
    {2018}
    {Cambridge in Advanced English (CAE)}
    {University of Cambridge}{}{}{\textit{Grade B}}

\section{\spa{Conocimientos}\en{Skills}}

\begin{itemize}
    
    \item \textbf{\spa{Experiencia con:}\en{Experience with:}:} Java, C, Python/Django, Javascript, Smalltalk, C\#/Unity, GDScript.
    \item \textbf{\spa{Otros lenguajes}\en{Other languages}:} HTML, CSS, Bootstrap, \LaTeX.
    \item \textbf{\spa{Programacion orientada a objetos.}\en{Object-oriented programming}}
    \item \textbf{\spa{Diagramación lógica, algoritmos y estructuras de datos.}\en{Algorithms and data structures}}
    \item \textbf{\spa{Bases de datos relacionales}\en{Relational databases}} / SQLite, MySQL.
    \item \textbf{\spa{Otras herramientas}\en{Other tools}:} Git, GNU Make, Linux, bash scripts.
    
\end{itemize}

\section{\spa{misc}\en{misc}}
\begin{itemize}
    \item \textbf{\spa{Inglés avanzado}\en{Advanced English}}
    \item \textbf{\spa{Español nativo}\en{Native Spanish}}
    \item \textbf{{\href{https://open.spotify.com/artist/1gmAvUNIqrPSGklxtoj514?si=zmeSuhggQwSlc_M0m4N1xA}{{\spa{hago música}\en{i make music}: }}(spotify: bokki (25.09M plays))}}
    \item {\href{https://github.com/maximosan/}{\textbf{Github: }https://github.com/maximosan/}}
    \item {\href{https://maximosantoro.com/}{\textbf{\spa{Página personal: }\en{Personal website: }}https://maximosantoro.com/}}
\end{itemize}

\end{document}
"
\ifdefined\inenglish
  \entrue
\else
  \spatrue
\fi

% Márgenes
\usepackage[scale=0.75, top=2cm, bottom=2.5cm]{geometry}

% Lastpage para el footer (x paginas de y)
% Fontawesome para iconos en la cabecera
% Xpatch para cambiar el formato del titulo
% Framed (environment shaded*) y textpos (environment textblock*) para el 'boton' de cambio de lenguaje
% Setspace para el interlineado
\usepackage{lastpage,fontawesome,xpatch,framed,setspace}
\usepackage[absolute,overlay]{textpos}

% Interlineado 1.5
\onehalfspacing

% Sacar el 'subject' del PDF (era 'Resume of...') y cambiar el titulo (para que no sea "nombre - titulo")
\makeatletter
\AtEndPreamble{\hypersetup{pdfsubject={},pdftitle={\@firstname~\@familyname}}}
\makeatother

% Comando de mes en español, para el footer
\newcommand{\MONTH}{%
  \ifcase\the\month
  \or \spa{Enero}\en{January} 
  \or \spa{Febrero}\en{February} 
  \or \spa{Marzo}\en{March} 
  \or \spa{Abril}\en{April} 
  \or \spa{Mayo}\en{May} 
  \or \spa{Junio}\en{June} 
  \or \spa{Julio}\en{July} 
  \or \spa{Agosto}\en{August} 
  \or \spa{Septiembre}\en{September} 
  \or \spa{Octubre}\en{October} 
  \or \spa{Noviembre}\en{November} 
  \or \spa{Diciembre}\en{December} 
  \fi
}

% Cambiar formato de título (en vez de 'Autor | Curriculum' ahora es 'Autor \n Curriculum')
\makeatletter
\xpatchcmd{\makehead}{\titlestyle{~|~\@title}}{\par\vskip1ex\titlestyle{\@title}}{}{}
\makeatother

% Achicar el subtítulo ('Curriculum Vitae')
\renewcommand*{\titlefont}{\fontsize{21}{25}\mdseries\upshape} 

% Header y Footer
\makeatletter
\fancyhead[R]{\color{gray}\@firstname{}~\@familyname{}}
\ifnum\value{page} > 1
  \fancyfoot[R]{\color{gray}\textit{\MONTH \the\year \\ \thepage/\pageref*{LastPage}}}
\else
  \fancyfoot[R]{\color{gray}\textit{\MONTH \the\year}}
\fi
\makeatother

% Solo footer para la primera pagina
\fancypagestyle{onlyfooter}{
\fancyhf{}
\ifnum\value{page} > 1
  \fancyfoot[R]{\color{gray}\textit{\MONTH \the\year \\ \thepage/\pageref*{LastPage}}}
\else
  \fancyfoot[R]{\color{gray}\textit{\MONTH \the\year}}
\fi
}

% Cambiar color de boton de lenguajes. Mientras mas cerca de 1, mas claro.
\definecolor{shadecolor}{gray}{0.9}


